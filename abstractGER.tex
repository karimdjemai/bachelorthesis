\chapter*{Zusammenfassung}
Eine der größten Hürden im Umgang mit moderner \textquote{virtual reality} (VR) Technologie ist die Begrenzung des begehbaren Bereiches (Trackingspace). Dies entsteht zum einen, durch räumlich limitierte Erfassung (auf Englisch: tracking) der Position und Rotation des Headsets und der Controller bei einigen Technologien (zum Beispiel den Modellen der \textquote{HTC VIVE}-Produktreihe
%TODO:quelle%
), zum anderen durch die Raumgröße der meisten VR-Setups. Es gibt unterschiedliche Lösungsansätze für dieses Problem. In dem hier vorgestellten Experiment werden drei davon miteinander Verglichen. Zum einen konventionelle Fortbewegungsarten für VR (Joystick-Steuerung und Teleportieren), zum anderen eine verhältnismäßig neue Fortbewegungsart mit dem Namen \textquote{Redirected Walking}. Hierbei handelt es sich um einen Sammelbegriff für Methoden bei denen die Nutzer:in durch subtile Änderungen an der VR-Fortbewegung durch den realen Trackingspace gelenkt wird, dabei aber die Illusion aufrecht erhalten wird, sie würde sich ohne Änderung fortbewegen. Diese Methoden lassen sich auf unterschiedliche Weisen kombinieren und sind so dem Kontext flexibel anpassbar. Diese Arbeit untersucht dabei einen konkreten Kontext: Automatisch generierte Level. Durch die Kombination von so genannten \textquote{Rotational Gains} und sogenannten \textquote{Impossible Spaces} und die Inkorporation davon in einen Levelgenerirungsalgorithmus ist es mir gelungen, theoretisch endlos große Level zu erschaffen, durch die sich die Nutzer:in fortbewegen kann obwohl sie sich nur in einem Begrenzt großen Trackingspace befindet. Diese Arbeit stellt ein experiment vor, das untersucht ob das räumliche Verständnis der Nutzer:innen mit dieser Fortbewegungsart beim durchschreiten des Levels besser gebildet wird, und ob da Präsenz Gefühl in der virtuellen Realität höher ist, als mit den herkömmlicheren Fortbewegungsmethoden.

%TODO: Studiendesign zusammenfassen

%TODO: Ergebnisse zusammenfassen

