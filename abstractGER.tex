\chapter*{Zusammenfassung}
Eine der größten Hürden im Umgang mit moderner \textquote{virtual reality} (VR) Technologie ist die Begrenzung des begehbaren Bereiches (Trackingspace).
Dies entsteht zum einen durch räumlich limitierte Erfassung (auf Englisch: tracking) der Position und Rotation des Headsets und der Controller bei einigen Technologien (beispielsweise den Modellen der \textquote{HTC VIVE}-Produktreihe\footnote{\href{https://www.vive.com/}{https://www.vive.com/}}).
% Zum anderen durch die limitierung von leerstehendem und sicherem Bereich den Nutzer:innen aufbringen können um sich darin frei bewegen zu können.
Zum anderen durch die, für durschnittliche Nutzer:innen, nur begrenzt zur Verfügung stehende Menge von leerstehendem und sicherem Bereich, der nötig ist um sich darin frei bewegen zu können.
Es gibt unterschiedliche Lösungsansätze für dieses Problem.
Einen davon stellen die sogenannten \textquote{Redirection}-Techniken dar.
Hierbei handelt es sich um einen Sammelbegriff für Methoden bei denen die Nutzer:in durch subtile Änderungen an der visuellen Darstellung der virtuellen Welt den Eindruck bekommen kann, sich über die Grenzen des realen Trackingspaces hinausbewegen zu können und dies auch zu tun. Dabei bleibt sie innerhalb dieser Grenzen, es wird aber die Illusion aufrecht erhalten, sie würde sich frei und ohne Manipulation der Darstellungsart der virtuellen Welt fortbewegen. Diese Methoden lassen sich auf unterschiedliche Weisen kombinieren und sind so dem Kontext flexibel anpassbar.
Diese Arbeit untersucht dabei einen konkreten Kontext: Prozedural generierte Level. Durch die Kombination zweier dieser Redirection-Techniken, den so genannten \textquote{Rotationgains} und den sogenannten \textquote{Impossible Spaces}, und durch die Inkorporation davon in einen Levelgenerirungsalgorithmus ist es mir gelungen, eine Level-Generierungs-Methode zu definieren, mit der beliebig lange, theoretisch sogar endlos lange Level erschaffen werden können. Dabei erhalten Nutzer:innen immer wieder die Illusion einen Raum zu betreten, der eigentlich außerhalb der Grenzen ihres Trackingspaces liegt, ohne die realen Grenze des Trackingspaces zu überschreiten.
Die Thesis stellt zudem eine informelle Pilotierungsstudie vor, bei der Proband:innen verschieden lange Level, die mit der eben genannten Methode generiert wurden, mit verschiedenen Fortbewegungsarten durchschritten. Bei einer davon, dem natürlichen Gehen (Real-Walking) wurden dabei auch die der Methode zugehörigen Redirection-Methoden eingesetzt, bei den anderen beiden (Joystick-Steuerung und Teleportation) nicht. Dabei wurde untersucht ob die Proband:innen bei der ersten Versuchsbedingung die von ihnen fortbewegte Distanz im Nachhinein besser schätzen können, und ob ihr Präsenzgefühl höher war als bei den anderen beiden Versuchsbedingungen. Es konnten keine signifikanten Unterschiede festgestellt werden.

