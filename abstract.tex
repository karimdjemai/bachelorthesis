\chapter*{Abstract}
One of the greatest challenges in dealing with modern \textquote{virtual reality} (VR) technology is the limitation of the accessible area (tracking space). This is due to the limited tracking of the position and rotation of the headset and the controllers in some technologies (for example, the models of the \textquote{HTC VIVE} product line)
%TODO: quelle
, and also to the size of the space in most VR setups. There are different solutions to this problem. In the experiment presented here, three of them are compared. On the one hand, conventional methods of locomotion in VR (joystick control and teleportation), and on the other hand, a relatively new method of locomotion called \textquote{redirected walking}. This is an umbrella term for methods in which the user is guided through the real tracking space by subtle changes to the virtual locomotion, while maintaining the illusion that they are moving without any changes. These methods can be combined in different ways and thus are adaptable to the context. This work investigates a certain context: automatically generated levels. By combining so-called \textquote{rotational gains} and so-called \textquote{impossible spaces} and incorporating them into a level generation algorithm, I have succeeded in creating theoretically infinitely large levels through which the user can move even though she is located in a limited tracking space. This paper presents an experiment that investigates whether the user's spatial understanding forms itself better with this method of locomotion as the user moves through the level, and whether the sense of presence in virtual reality is higher, compared to the more traditional methods of locomotion. %TODO: Ergebnisse zusammenfassen