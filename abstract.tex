\chapter*{Abstract}
One of the greatest challenges in dealing with modern virtual reality (VR) technology is the limitation of the accessible area (tracking space).
This is caused by spatially limited tracking of the position and rotation of the headset and the controllers in some technologies (for example, the models of the \textquote{HTC VIVE} product line\footnote{\href{https://www.vive.com/}{https://www.vive.com/}})
And also because of limited amount of vacant and safe area the average user has, which is necessary to be able to move around freely.
There are different solutions to this problem.
One of which are the so called \textquote{redirection}-techniques.
This is an umbrella term for methods where subtle changes to the visual representation of the virtual environment are made, such that the user can move to areas of it that would otherwise be inaccessable because they would be located beyond the boundaries of the real tracking space. Meanwhile they actually stay inside those boundaries but the illusion they could move freely and without manipulation of the visual environment is kept.
These methods can be combined in different ways and thus are adaptable to the context. This work investigates a certain context: procedurally generated levels. By combining so-called \textquote{rotational gains} and so-called \textquote{impossible spaces} and incorporating them into a level generation algorithm, I have succeeded in creating a level generation method that can produce any length, theoretically even infinitely long levels.
The users move trough those levels with the repeated illusion of accessing a new room that would have previously been outside the boundaries of their real tracking space.
The paper presents an informal pilot study in which subjects walked through levels of different lengths generated by the above-mentioned method using different techniques of locomotion. For one of them, real walking, the redirection techniques associated with the method were used, for the other two (joystick control and teleportation) they were not. It was examined whether the test subjects in the real walking condition were better able to estimate the distance they had moved, and whether their sense of presence was higher than in the other two test conditions. No significant differences were found.